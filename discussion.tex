\section{Discussion (3 pages)}
\label{sec:discussion}

\subsection{Application to real world case}
\label{sec:discussion:higgs}

HOW APPLIED TO HIGGS ANALYSIS.

The actual $H \rightarrow \gamma\gamma$ analysis is significantly more
complex than the simplified version used here. In particular, the 2011 data
sample is split into five categories (which become nine for the 2012 data).
Because the categories (by definition) have different selection criteria,
they can have different background shapes.
There is no {\it a priori} reason to make any assumptions that the functions
used in each category should be the same. Hence, each category should be
tested with all functions, in a similar way to the above.

However, the major complication arises because there are common systematics 
across the categories, arising from nuisance parameters in the signal
model.
In the absence of these common nuisance parameters,
the different categories could be profiled independently, using the
minimum envelope technique to produce a curve per category. These could then
be summed to give the overall profile curve. However, with common
nuisance parameters, all categories must be profiled at the same time.
Since minimisation code to handle the 
discrete nuisance parameter identifying the
function seems difficult, in practical terms, this means that all possible
combinations of each function in each category must be fitted.
The minimum envelope made from the results of all these combinations would
then be found. While this is conceptually straightforward, the actual
implementation is probably prohibitive. For example, using the 16 functions
discussed above, then there would be $16^5 \sim$ 1M combinations of functions
to be fitted for the 2011 data (and $16^9 \sim$ 70B combinations for the
2012 data).

It happens that the correlations of the nuisance parameters
are relatively small.
It may be a good approximation 
to first identify the two or three most likely functions in each category
which will contribute to the minimum envelope, ignoring the correlations.
Only these functions would then be used in the full fit, giving at most
$3^5 \sim 200$ or $3^9 \sim$ 20k combinations. Studies into the 
implementation are ongoing.

\subsection{Envelope smoothing}
\label{sec:discussion:smoothing}

DISCRETE FUNCTIONS MEAN NON-SMOOTH ENVELOPE

APPLY SMOOTHING USING [SUM 2NLL EXP(-2NLL/2)]/[SUM EXP(-2NLL/2)]?


\subsection{Bayesian equivalent method}
\label{sec:discussion:bayesian}
