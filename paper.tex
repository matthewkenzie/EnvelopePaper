\documentclass[11pt,a4paper]{article}
\usepackage{graphicx}
\usepackage{rotating}

\oddsidemargin=0pt           % No extra space wasted after first inch.
\evensidemargin=0pt          % Ditto (for two-sided output).
\topmargin=0pt               % Same for top of page.
\headheight=0pt              % Don't waste any space on unused headers.
\headsep=0pt
\textwidth=16cm              % Effectively controls the right margin.
\textheight=24cm

\begin{document}
\title{
\Large \bf Handling uncertainties in background shapes: the envelope method}
\author{P.~D.~Dauncey$^1$, G.~J.~Davies$^1$, M.~''Sandy''~Kenzie$^2$, N.~Wardle$^2$\\
$^1$Imperial College London, $^2$CERN}

\maketitle

\begin{abstract}
A common problem in data analysis is measuring a narrow signal on a smoothly
varying background. In cases where the background functional form is not
{\it a priori} known, then some extra uncertainty must be assigned to the
signal parameters because of this lack of knowledge.
A method for assigning an error from this cause is presented. It is based on
treating the function uncertainty as a discrete nuisance parameter and finding
an ``envelope'' which encompasses the lowest log-likelihood values for any given
signal parameter. The bias and coverage of this method are shown to be good.
\end{abstract}

\tableofcontents
\newpage

\input introduction.tex
\input concept.tex
\input functions.tex
\input correction.tex
\input discussion.tex
\input conclusions.tex

\section{Ackowledgements}
We thank Chris Seez and Louis Lyons for informative discussions.
This work was partially supported by the Science and Technology Facilities
Council, UK.

\bibliographystyle{unsrt}
\bibliography{paper}
%\documentclass[11pt,a4paper]{article}
\usepackage{graphicx}
\usepackage{rotating}

\oddsidemargin=0pt           % No extra space wasted after first inch.
\evensidemargin=0pt          % Ditto (for two-sided output).
\topmargin=0pt               % Same for top of page.
\headheight=0pt              % Don't waste any space on unused headers.
\headsep=0pt
\textwidth=16cm              % Effectively controls the right margin.
\textheight=24cm

\begin{document}
\title{
\Large \bf Handling uncertainties in background shapes: the envelope method}
\author{P.~D.~Dauncey$^1$, G.~J.~Davies$^1$, M.~Kenzie$^2$, N.~Wardle$^2$\\
$^1$Imperial College London, $^2$CERN}

\maketitle

\begin{abstract}
A common problem in data analysis is measuring a narrow signal on a smoothly
varying background. In cases where the background functional form is not
{\it a priori} known, then some extra uncertainty must be assigned to the
signal parameters because of this lack of knowledge.
A method for assigning an error from this cause is presented. It is based on
treating the function uncertainty as a discrete nuisance parameter and finding
an ``envelope'' which encompasses the lowest log-likelihood values for any given
signal parameter. The bias and coverage of this method are shown to be good.
\end{abstract}

\tableofcontents
\newpage

\input introduction.tex
\input concept.tex
\input functions.tex
\input correction.tex
\input discussion.tex
\input conclusions.tex

\section{Ackowledgements}
We thank Louis Lyons for informative discussions. This work was partially
supported by STFC, UK.

\input references.tex

\end{document}


\end{document}
